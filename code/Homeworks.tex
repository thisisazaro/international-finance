% Options for packages loaded elsewhere
\PassOptionsToPackage{unicode}{hyperref}
\PassOptionsToPackage{hyphens}{url}
\documentclass[
]{article}
\usepackage{xcolor}
\usepackage[margin=1in]{geometry}
\usepackage{amsmath,amssymb}
\setcounter{secnumdepth}{5}
\usepackage{iftex}
\ifPDFTeX
  \usepackage[T1]{fontenc}
  \usepackage[utf8]{inputenc}
  \usepackage{textcomp} % provide euro and other symbols
\else % if luatex or xetex
  \usepackage{unicode-math} % this also loads fontspec
  \defaultfontfeatures{Scale=MatchLowercase}
  \defaultfontfeatures[\rmfamily]{Ligatures=TeX,Scale=1}
\fi
\usepackage{lmodern}
\ifPDFTeX\else
  % xetex/luatex font selection
\fi
% Use upquote if available, for straight quotes in verbatim environments
\IfFileExists{upquote.sty}{\usepackage{upquote}}{}
\IfFileExists{microtype.sty}{% use microtype if available
  \usepackage[]{microtype}
  \UseMicrotypeSet[protrusion]{basicmath} % disable protrusion for tt fonts
}{}
\makeatletter
\@ifundefined{KOMAClassName}{% if non-KOMA class
  \IfFileExists{parskip.sty}{%
    \usepackage{parskip}
  }{% else
    \setlength{\parindent}{0pt}
    \setlength{\parskip}{6pt plus 2pt minus 1pt}}
}{% if KOMA class
  \KOMAoptions{parskip=half}}
\makeatother
\usepackage{graphicx}
\makeatletter
\newsavebox\pandoc@box
\newcommand*\pandocbounded[1]{% scales image to fit in text height/width
  \sbox\pandoc@box{#1}%
  \Gscale@div\@tempa{\textheight}{\dimexpr\ht\pandoc@box+\dp\pandoc@box\relax}%
  \Gscale@div\@tempb{\linewidth}{\wd\pandoc@box}%
  \ifdim\@tempb\p@<\@tempa\p@\let\@tempa\@tempb\fi% select the smaller of both
  \ifdim\@tempa\p@<\p@\scalebox{\@tempa}{\usebox\pandoc@box}%
  \else\usebox{\pandoc@box}%
  \fi%
}
% Set default figure placement to htbp
\def\fps@figure{htbp}
\makeatother
\setlength{\emergencystretch}{3em} % prevent overfull lines
\providecommand{\tightlist}{%
  \setlength{\itemsep}{0pt}\setlength{\parskip}{0pt}}
\usepackage{bookmark}
\IfFileExists{xurl.sty}{\usepackage{xurl}}{} % add URL line breaks if available
\urlstyle{same}
\hypersetup{
  pdftitle={Homeworks},
  pdfauthor={Iushin Nikolai},
  hidelinks,
  pdfcreator={LaTeX via pandoc}}

\title{Homeworks}
\author{Iushin Nikolai}
\date{2025-04-27}

\begin{document}
\maketitle

{
\setcounter{tocdepth}{2}
\tableofcontents
}
\section{Homework 2}\label{homework-2}

\subsection{Q1. CES Utility}\label{q1.-ces-utility}

\textbf{Given}\\
\[
u = \Bigl(\gamma\,y_{1}^{\frac{\sigma-1}{\sigma}} + y_{2}^{\frac{\sigma-1}{\sigma}}\Bigr)^{\frac{\sigma}{\sigma-1}},
\qquad
\text{budget: }p_{1}y_{1} + p_{2}y_{2} = I.
\]

\textbf{Find} the ratio \(y_{1}/y_{2}\) that maximizes \(u\).

\textbf{Solution}\\
1. Define the intermediate function\\
\[
   V = \gamma\,y_{1}^{\frac{\sigma-1}{\sigma}} + y_{2}^{\frac{\sigma-1}{\sigma}}
   \] and set up the Lagrangian\\
\[
   \mathcal{L} = V + \lambda\,(I - p_{1}y_{1} - p_{2}y_{2}).
   \] 2. First‐order conditions:\\
\[
   \frac{\partial\mathcal{L}}{\partial y_{1}}
   = \gamma\,\frac{\sigma-1}{\sigma}\,y_{1}^{\frac{\sigma-1}{\sigma}-1} - \lambda\,p_{1} = 0,
   \quad
   \frac{\partial\mathcal{L}}{\partial y_{2}}
   = \frac{\sigma-1}{\sigma}\,y_{2}^{\frac{\sigma-1}{\sigma}-1} - \lambda\,p_{2} = 0.
   \] 3. Divide the two equations to eliminate \(\lambda\):\\
\[
   \frac{\gamma\,\frac{\sigma-1}{\sigma}\,y_{1}^{\frac{\sigma-1}{\sigma}-1}}{\frac{\sigma-1}{\sigma}\,y_{2}^{\frac{\sigma-1}{\sigma}-1}}
   = \frac{p_{1}}{p_{2}}
   \;\Longrightarrow\;
   \gamma\,y_{1}^{-\tfrac{1}{\sigma}}\,p_{2}
   = y_{2}^{-\tfrac{1}{\sigma}}\,p_{1}
   \;\Longrightarrow\;
   \Bigl(\frac{y_{1}}{y_{2}}\Bigr)^{\!\tfrac{1}{\sigma}}
   = \frac{\gamma\,p_{2}}{p_{1}}.
   \] 4. Therefore, the optimal ratio is\\
\[
   \boxed{
     \frac{y_{1}}{y_{2}}
     = \biggl(\gamma\,\frac{p_{2}}{p_{1}}\biggr)^{\!\sigma}
   }.
   \]

\subsection{Q2. Cobb--Douglas in an Open
Economy}\label{q2.-cobbdouglas-in-an-open-economy}

\textbf{Given}\\
\[
U(y_{1},y_{2}) = y_{1}^{\alpha}\,y_{2}^{\beta},\quad \alpha + \beta = 1,
\] endowment \((y_{1}^{e},y_{2}^{e})\), world prices \((p_{1},p_{2})\).

\begin{enumerate}
\def\labelenumi{\arabic{enumi}.}
\tightlist
\item
  \textbf{Income} (value of endowment):\\
  \[
  I = p_{1}\,y_{1}^{e} + p_{2}\,y_{2}^{e}.
  \]
\item
  \textbf{Budget constraint}\\
  \[
  p_{1}\,y_{1} + p_{2}\,y_{2} = I.
  \]
\item
  \textbf{Maximization} yields standard expenditure shares:\\
  \[
  p_{1}\,y_{1} = \alpha\,I,
  \quad
  p_{2}\,y_{2} = \beta\,I.
  \]
\item
  \textbf{Demand functions}:\\
  \[
  \boxed{
    y_{1}^{*} = \frac{\alpha\,I}{p_{1}},
    \quad
    y_{2}^{*} = \frac{\beta\,I}{p_{2}}
  }.
  \]
\item
  Substituting \(I = p_{1}\,y_{1}^{e} + p_{2}\,y_{2}^{e}\) gives:\\
  \[
  y_{1}^{*}
  = \alpha\Bigl(y_{1}^{e} + \tfrac{p_{2}}{p_{1}}\,y_{2}^{e}\Bigr),
  \quad
  y_{2}^{*}
  = \beta\Bigl(\tfrac{p_{1}}{p_{2}}\,y_{1}^{e} + y_{2}^{e}\Bigr).
  \]
\end{enumerate}

\section{Homework 3 -- Equivalence of the Two
Equilibria}\label{homework-3-equivalence-of-the-two-equilibria}

\textbf{Claim.} The home country's equilibrium under\\
1. ``Separate free-trade'' (Situation 1) and\\
2. ``Integrated world economy'' (Situation 2)

coincide. In both cases the relative price \(p\) is the same, so home
produces and consumes the same bundle.

\subsection{Definitions}\label{definitions}

\begin{itemize}
\tightlist
\item
  Let\\
  \[
    RS_H(p) \;=\;\text{home’s relative‐supply of good 1 (units of 1 per 2) at price }p,
  \] \[
    RS_F(p) \;=\;\text{foreign’s relative‐supply of good 1 at price }p,
  \] \[
    RS_W(p)\;=\;RS_H(p)+RS_F(p)\quad\text{(world relative-supply).}
  \]
\item
  Let \(RD(p)\) be world relative-demand (units of 1 per 2) at price
  \(p\).
\end{itemize}

\subsection{Situation 1: Separate Free
Trade}\label{situation-1-separate-free-trade}

\begin{enumerate}
\def\labelenumi{\arabic{enumi}.}
\tightlist
\item
  \textbf{World price determination.}\\
  Since both countries take the same world price \(p\) and supply goods
  competitively, the \textbf{aggregate equilibrium} price \(p^*\) solves
  \[
  RS_W(p^*) \;=\; RD(p^*)\,.
  \]
\item
  \textbf{Country‐specific outcome.}\\
  At \(p^*\), home chooses its profit‐maximizing production point on its
  PPF and its utility‐maximizing consumption on its indifference map.
  Call this bundle \(\bigl(y_1^H,y_2^H\bigr)\).
\end{enumerate}

\subsection{Situation 2: Integrated World
Economy}\label{situation-2-integrated-world-economy}

\begin{enumerate}
\def\labelenumi{\arabic{enumi}.}
\tightlist
\item
  \textbf{Aggregate economy.}\\
  Treat home + foreign as one big country with relative‐supply\\
  \[
    RS_W(p)=RS_H(p)+RS_F(p).
  \]
\item
  \textbf{World equilibrium.}\\
  The integrated economy's equilibrium price \(\tilde p\) solves the
  \textbf{same} equation \[
  RS_W(\tilde p) \;=\; RD(\tilde p).
  \]
\item
  \textbf{Specialization assumption.}\\
  At \(\tilde p\) (between the two autarky prices), home fully
  specializes in good 1 and foreign in good 2, but the determining
  equation for \(\tilde p\) is identical.
\end{enumerate}

\subsection{Conclusion}\label{conclusion}

\begin{itemize}
\tightlist
\item
  Both situations pick the \textbf{same} relative price:\\
  \[
    p^* = \tilde p.
  \]
\item
  Home faces the same price in both cases, so by price‐taking behavior
  its production \& consumption plan \(\bigl(y_1^H,y_2^H\bigr)\) is
  \textbf{identical}.
\item
  \textbf{Therefore} the two equilibria coincide for the home country.
\end{itemize}

\subsection{Graphical Intuition}\label{graphical-intuition}

\begin{itemize}
\tightlist
\item
  In both pictures you draw the world relative-supply curve \(RS_W(p)\)
  and the relative-demand curve \(RD(p)\).
\item
  The intersection \(RS_W=RD\) is \textbf{exactly} the same point in
  Situation 1 and Situation 2.
\item
  Home then ``picks off'' its point on its own PPF at that price---so
  you end up at the same coordinate in both diagrams.
\end{itemize}

\(\boxed{\text{Hence the equilibria are equal.}}\)

\section{Homework 4}\label{homework-4}

\subsection{\texorpdfstring{Q1. Equilibrium in DFS (1977) with
\(U=\int_{0}^{1}b(z)\ln c(z)\,dz\)}{Q1. Equilibrium in DFS (1977) with U=\textbackslash int\_\{0\}\^{}\{1\}b(z)\textbackslash ln c(z)\textbackslash,dz}}\label{q1.-equilibrium-in-dfs-1977-with-uint_01bzln-czdz}

\begin{enumerate}
\def\labelenumi{\arabic{enumi}.}
\tightlist
\item
  \textbf{Preferences \& demand}

  \begin{itemize}
  \tightlist
  \item
    Utility:\\
    \[
      U=\int_{0}^{1}b(z)\ln c(z)\,dz,\quad \int_{0}^{1}b(z)\,dz=1.
    \]
  \item
    FOC w.r.t.~\(c(z)\):\\
    \[
      \frac{b(z)}{c(z)}=\lambda\,p(z)
      \;\;\Longrightarrow\;\;
      c(z)=\frac{b(z)}{\lambda\,p(z)}.
    \]
  \item
    Budget \(\int_{0}^{1}p(z)c(z)\,dz=I\) implies \(\lambda=1/I\), so \[
      c(z)=I\,\frac{b(z)}{p(z)}.
    \]
  \item
    \textbf{Expenditure share} on good \(z\):\\
    \(\,p(z)\,c(z)=I\,b(z)\) → ``Cobb--Douglas'' across the continuum.
  \end{itemize}
\item
  \textbf{Production \& zero profit}

  \begin{itemize}
  \tightlist
  \item
    Home has unit‐labour requirements \(a(z)\), foreign \(a^*(z)\).\\
  \item
    Let relative wage \(\omega=w/w^*\).\\
  \item
    Zero‐profit prices: \[
      p(z)=
      \begin{cases}
        w\,a(z),& z<z^*,\\
        w^*\,a^*(z),& z>z^*,
      \end{cases}
    \] where the \textbf{cutoff} \(z^*\) solves \[
      w\,a(z^*)=w^*\,a^*(z^*)
      \quad\Longrightarrow\quad
      \omega=\frac{a^*(z^*)}{a(z^*)}.
    \]
  \end{itemize}
\item
  \textbf{World relative‐supply \& demand}

  \begin{itemize}
  \tightlist
  \item
    \textbf{Supply} of home goods (z ∈ {[}0,z*{]}) vs.~foreign goods (z
    ∈ {[}z*,1{]}):\\
    \[
      RS(\omega)
      =\frac{L\int_{0}^{z^*}\!1/a(z)\,dz}
            {L^*\int_{z^*}^{1}\!1/a^*(z)\,dz}.
    \]
  \item
    \textbf{Demand} for home goods: \[
      RD(\omega)
      =\frac{\displaystyle\int_{0}^{z^*}\!b(z)\,dz}
            {\displaystyle\int_{z^*}^{1}\!b(z)\,dz}.
    \]
  \item
    \textbf{Equilibrium} requires simultaneously \[
      RS(\omega)=RD(\omega),
      \quad
      \omega=\frac{a^*(z^*)}{a(z^*)}.
    \]
  \end{itemize}
\item
  \textbf{Effect of \(b(z)\)}

  \begin{itemize}
  \tightlist
  \item
    Only the \textbf{shape} of the relative‐demand curve \(RD(\omega)\)
    changes (weights \(b(z)\) instead of uniform).\\
  \item
    The \textbf{qualitative} features---existence of a unique cutoff
    \(z^*\), full specialization on each side---remain intact.
  \end{itemize}
\end{enumerate}

\subsection{\texorpdfstring{Q2. Intuitive transition when
\(L^*\to L^{*\prime}\)}{Q2. Intuitive transition when L\^{}*\textbackslash to L\^{}\{*\textbackslash prime\}}}\label{q2.-intuitive-transition-when-lto-lprime}

\begin{enumerate}
\def\labelenumi{\arabic{enumi}.}
\tightlist
\item
  \textbf{Foreign labor rises ⇒ supply shift}

  \begin{itemize}
  \tightlist
  \item
    Larger \(L^*\) raises foreign output of every good ⇒ world
    \textbf{relative supply} of home goods falls.
  \end{itemize}
\item
  \textbf{Relative‐price adjustment}

  \begin{itemize}
  \tightlist
  \item
    To clear the market, the relative price \(\omega=w/w^*\) must
    \textbf{decline}.\\
  \item
    Graphically, the supply curve \(A(z)\) shifts so its intersection
    with demand \(B(z)\) moves from \(E\) to \(E'\).
  \end{itemize}
\item
  \textbf{New cutoff \(z^*\)}

  \begin{itemize}
  \tightlist
  \item
    Lower \(\omega\) solves \(\omega=a^*(z^*)/a(z^*)\) at a
    \textbf{smaller} \(z^*\).\\
  \item
    Home now specializes in a narrower set of goods (only the
    lowest-\(z\) range).
  \end{itemize}
\item
  \textbf{Partial-equilibrium analogy}

  \begin{itemize}
  \tightlist
  \item
    Just like ↑ supply in a single market → ↓ price + new quantity, here
    ↑ foreign labor endowment → ↓ relative price of home goods +
    adjusted specialization until the new intersection \(E'\) clears the
    world market.
  \end{itemize}
\end{enumerate}

\section{Homework 5}\label{homework-5}

\textbf{Answer.} Since \[
a_i(j)\sim\mathrm{Weibull}\bigl(\text{shape}=\alpha_i,\;\text{scale}=\tfrac1{\lambda_i}\bigr)
\] and \[
p_{ni}(j)=\frac{w_i}{d_{ni}}\;a_i(j),
\] it follows that \(p_{ni}(j)\) is also Weibull with

\begin{itemize}
\tightlist
\item
  \textbf{Shape parameter:} \(\alpha_i\)\\
\item
  \textbf{Scale parameter:}
  \(\displaystyle \frac{w_i}{d_{ni}\,\lambda_i}\)
\end{itemize}

Its CDF is \[
\begin{aligned}
F_{p}(x)
&= \Pr\bigl(p_{ni}\le x\bigr)
= \Pr\Bigl(a_i \le \tfrac{d_{ni}}{w_i}\,x\Bigr)\\
&=1-\exp\!\Bigl[-\bigl(\lambda_i\cdot\tfrac{d_{ni}}{w_i}\,x\bigr)^{\alpha_i}\Bigr].
\end{aligned}
\]

\section{Homework 6}\label{homework-6}

Let\\
\[
V\bigl(y_{1};L,K\bigr)
\;=\;
\max_{\substack{L_{1}+L_{2}\le L\\K_{1}+K_{2}\le K\\f_{1}(L_{1},K_{1})\ge y_{1}}}
f_{2}(L_{2},K_{2})
\] be the PPF viewed as a function of endowments \((L,K)\) at fixed
\(y_{1}\). Then:

\begin{enumerate}
\def\labelenumi{\arabic{enumi}.}
\item
  \textbf{Envelope theorem}\\
  At the optimum, \[
  \frac{\partial V}{\partial L}
  = \lambda_{L},
  \quad
  \frac{\partial V}{\partial K}
  = \lambda_{K},
  \] where \(\lambda_{L},\lambda_{K}\) are the Lagrange multipliers on
  the labor and capital constraints.
\item
  \textbf{CRS ⇒ homogeneity}\\
  Because each sector's technology is CRS, \[
  \lambda_{L}\,L \;+\;\lambda_{K}\,K
  \;=\;
  f_{2}(L_{2}^{*},K_{2}^{*})
  \;=\;
  V(y_{1};L,K).
  \]
\item
  \textbf{Tripling endowments}\\
  Take \((L,K)\to(3L,3K)\). Then by the envelope theorem and
  homogeneity, \[
  V(y_{1};3L,3K)
  -V(y_{1};L,K)
  =2\bigl(\lambda_{L}L+\lambda_{K}K\bigr)
  =2\,V(y_{1};L,K).
  \] Hence \[
    V(y_{1};3L,3K)
    =3\,V(y_{1};L,K).
  \]
\item
  \textbf{Conclusion: outward shift}\\
  For every \(y_{1}\), the maximum feasible \(y_{2}\) triples.
  Equivalently, \[
    \boxed{
      \text{New PPF: }
      y_{2}^{\rm new}(y_{1})
      =3\,y_{2}^{\rm old}(y_{1})
    }.
  \] The frontier simply scales out by a factor of 3, with its slope at
  each point unchanged.
\end{enumerate}

\end{document}

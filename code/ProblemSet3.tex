% Options for packages loaded elsewhere
\PassOptionsToPackage{unicode}{hyperref}
\PassOptionsToPackage{hyphens}{url}
\documentclass[
]{article}
\usepackage{xcolor}
\usepackage[margin=1in]{geometry}
\usepackage{amsmath,amssymb}
\setcounter{secnumdepth}{5}
\usepackage{iftex}
\ifPDFTeX
  \usepackage[T1]{fontenc}
  \usepackage[utf8]{inputenc}
  \usepackage{textcomp} % provide euro and other symbols
\else % if luatex or xetex
  \usepackage{unicode-math} % this also loads fontspec
  \defaultfontfeatures{Scale=MatchLowercase}
  \defaultfontfeatures[\rmfamily]{Ligatures=TeX,Scale=1}
\fi
\usepackage{lmodern}
\ifPDFTeX\else
  % xetex/luatex font selection
\fi
% Use upquote if available, for straight quotes in verbatim environments
\IfFileExists{upquote.sty}{\usepackage{upquote}}{}
\IfFileExists{microtype.sty}{% use microtype if available
  \usepackage[]{microtype}
  \UseMicrotypeSet[protrusion]{basicmath} % disable protrusion for tt fonts
}{}
\makeatletter
\@ifundefined{KOMAClassName}{% if non-KOMA class
  \IfFileExists{parskip.sty}{%
    \usepackage{parskip}
  }{% else
    \setlength{\parindent}{0pt}
    \setlength{\parskip}{6pt plus 2pt minus 1pt}}
}{% if KOMA class
  \KOMAoptions{parskip=half}}
\makeatother
\usepackage{longtable,booktabs,array}
\usepackage{calc} % for calculating minipage widths
% Correct order of tables after \paragraph or \subparagraph
\usepackage{etoolbox}
\makeatletter
\patchcmd\longtable{\par}{\if@noskipsec\mbox{}\fi\par}{}{}
\makeatother
% Allow footnotes in longtable head/foot
\IfFileExists{footnotehyper.sty}{\usepackage{footnotehyper}}{\usepackage{footnote}}
\makesavenoteenv{longtable}
\usepackage{graphicx}
\makeatletter
\newsavebox\pandoc@box
\newcommand*\pandocbounded[1]{% scales image to fit in text height/width
  \sbox\pandoc@box{#1}%
  \Gscale@div\@tempa{\textheight}{\dimexpr\ht\pandoc@box+\dp\pandoc@box\relax}%
  \Gscale@div\@tempb{\linewidth}{\wd\pandoc@box}%
  \ifdim\@tempb\p@<\@tempa\p@\let\@tempa\@tempb\fi% select the smaller of both
  \ifdim\@tempa\p@<\p@\scalebox{\@tempa}{\usebox\pandoc@box}%
  \else\usebox{\pandoc@box}%
  \fi%
}
% Set default figure placement to htbp
\def\fps@figure{htbp}
\makeatother
\setlength{\emergencystretch}{3em} % prevent overfull lines
\providecommand{\tightlist}{%
  \setlength{\itemsep}{0pt}\setlength{\parskip}{0pt}}
\usepackage{bookmark}
\IfFileExists{xurl.sty}{\usepackage{xurl}}{} % add URL line breaks if available
\urlstyle{same}
\hypersetup{
  pdftitle={Problem Set 3},
  pdfauthor={Iushin Nikolai},
  hidelinks,
  pdfcreator={LaTeX via pandoc}}

\title{Problem Set 3}
\author{Iushin Nikolai}
\date{2025-04-16}

\begin{document}
\maketitle

{
\setcounter{tocdepth}{2}
\tableofcontents
}
\section{Problem Set 3}\label{problem-set-3}

\subsection{Task 1: Solow Growth Model}\label{task-1-solow-growth-model}

\textbf{Given:} - Production function: \(Y_t = A \cdot K_t^{0.4}\),
initially \(A = 1\) - Saving rate: \(s = 0.4\) - Depreciation rate:
\(\delta = 0.1\) - Initial capital stock: \(K_1 = 4\)

\subsubsection{Capital Accumulation
Equation}\label{capital-accumulation-equation}

The general capital accumulation equation is: \[
K_{t+1} = (1 - \delta)K_t + sY_t
\] Substituting the production function \(Y_t = A \cdot K_t^{0.4}\) and
\(A = 1\): \[
K_{t+1} = 0.9K_t + 0.4 \cdot K_t^{0.4}
\]

\subsubsection{\texorpdfstring{Compute Capital \(K_t\) for \(t = 2\) to
\(t = 5\)}{Compute Capital K\_t for t = 2 to t = 5}}\label{compute-capital-k_t-for-t-2-to-t-5}

Using the formula from (1.1.1):

\begin{itemize}
\tightlist
\item
  \(K_1 = 4\)
\end{itemize}

\textbf{Step 1: \(t = 2\)}\\
\[
K_2 = 0.9 \cdot 4 + 0.4 \cdot 4^{0.4} \approx 3.6 + 0.4 \cdot 1.740 = 3.6 + 0.696 = 4.296
\]

\textbf{Step 2: \(t = 3\)}\\
\[
K_3 = 0.9 \cdot 4.296 + 0.4 \cdot 4.296^{0.4} \approx 3.866 + 0.716 = 4.582
\]

\textbf{Step 3: \(t = 4\)}\\
\[
K_4 = 0.9 \cdot 4.582 + 0.4 \cdot 4.582^{0.4} \approx 4.124 + 0.732 = 4.856
\]

\textbf{Step 4: \(t = 5\)}\\
\[
K_5 = 0.9 \cdot 4.856 + 0.4 \cdot 4.856^{0.4} \approx 4.370 + 0.746 = 5.116
\]

\subsubsection{\texorpdfstring{Compute Output \(Y_t\) for \(t = 2\) to
\(t = 5\)}{Compute Output Y\_t for t = 2 to t = 5}}\label{compute-output-y_t-for-t-2-to-t-5}

Use the production function: \(Y_t = K_t^{0.4}\)

\begin{itemize}
\tightlist
\item
  \(Y_2 = 4.296^{0.4} \approx 1.789\)
\item
  \(Y_3 = 4.582^{0.4} \approx 1.829\)
\item
  \(Y_4 = 4.856^{0.4} \approx 1.864\)
\item
  \(Y_5 = 5.116^{0.4} \approx 1.892\)
\end{itemize}

\subsubsection{\texorpdfstring{Technological Progress:
\(A = 1.5\)}{Technological Progress: A = 1.5}}\label{technological-progress-a-1.5}

Now the production function becomes:\\
\[
Y_t = 1.5 \cdot K_t^{0.4}
\]

New capital accumulation equation: \[
K_{t+1} = 0.9K_t + 0.6 \cdot K_t^{0.4}
\]

\textbf{Recalculate \(K_t\) and \(Y_t\)}

\textbf{Step 1: \(t = 2\)}\\
\[
K_2 = 0.9 \cdot 4 + 0.6 \cdot 4^{0.4} \approx 3.6 + 1.044 = 4.644
\]

\textbf{Step 2: \(t = 3\)}\\
\[
K_3 = 0.9 \cdot 4.644 + 0.6 \cdot 4.644^{0.4} \approx 4.180 + 1.096 = 5.276
\]

\textbf{Step 3: \(t = 4\)}\\
\[
K_4 = 0.9 \cdot 5.276 + 0.6 \cdot 5.276^{0.4} \approx 4.748 + 1.131 = 5.879
\]

\textbf{Step 4: \(t = 5\)}\\
\[
K_5 = 0.9 \cdot 5.879 + 0.6 \cdot 5.879^{0.4} \approx 5.291 + 1.160 = 6.451
\]

\textbf{Output with technology \(A = 1.5\)}

\begin{itemize}
\tightlist
\item
  \(Y_2 = 1.5 \cdot 4.644^{0.4} \approx 2.741\)
\item
  \(Y_3 = 1.5 \cdot 5.276^{0.4} \approx 2.827\)
\item
  \(Y_4 = 1.5 \cdot 5.879^{0.4} \approx 2.899\)
\item
  \(Y_5 = 1.5 \cdot 6.451^{0.4} \approx 2.957\)
\end{itemize}

\textbf{Summary Tables}

\textbf{Without Technological Progress:}

\begin{longtable}[]{@{}lll@{}}
\toprule\noalign{}
Period (t) & Capital \(K_t\) & Output \(Y_t\) \\
\midrule\noalign{}
\endhead
\bottomrule\noalign{}
\endlastfoot
2 & 4.296 & 1.789 \\
3 & 4.582 & 1.829 \\
4 & 4.856 & 1.864 \\
5 & 5.116 & 1.892 \\
\end{longtable}

\textbf{With Technological Progress (A = 1.5):}

\begin{longtable}[]{@{}lll@{}}
\toprule\noalign{}
Period (t) & Capital \(K_t\) & Output \(Y_t\) \\
\midrule\noalign{}
\endhead
\bottomrule\noalign{}
\endlastfoot
2 & 4.644 & 2.741 \\
3 & 5.276 & 2.827 \\
4 & 5.879 & 2.899 \\
5 & 6.451 & 2.957 \\
\end{longtable}

\begin{center}\rule{0.5\linewidth}{0.5pt}\end{center}

\subsection{Task 2: Consumption Under Tax Policy (3-Period
Model)}\label{task-2-consumption-under-tax-policy-3-period-model}

\textbf{Given:} - Utility function: \(u(c) = c - \frac{1}{100}c^2\) -
Time horizon: 3 periods, \(t = 1, 2, 3\) - Income:
\(y_1 = y_2 = y_3 = 20\) - No interest rate or discounting assumed
(implied by problem unless stated otherwise)

\subsubsection{If the consumer chooses to
consume:}\label{if-the-consumer-chooses-to-consume}

\begin{itemize}
\tightlist
\item
  \(c_1 = 15\)
\item
  \(c_2 = 20\)
\item
  \(c_3 = 20\)
\end{itemize}

\textbf{Find: Savings \(s_1, s_2, s_3\)}

Savings in each period is: \[
s_t = y_t - c_t \text{ (can be negative)}
\]

\begin{itemize}
\tightlist
\item
  \(s_1 = 20 - 15 = 5\)
\item
  \(s_2 = 20 - 20 = 0\)
\item
  \(s_3 = 20 - 20 = 0\)
\end{itemize}

\textbf{Answer:} - \(s_1 = 5\), \(s_2 = 0\), \(s_3 = 0\)

\subsubsection{Income tax of 20\% in period 1
only}\label{income-tax-of-20-in-period-1-only}

New income: - \(y_1 = 20 \times (1 - 0.2) = 16\) - \(y_2 = 20\),
\(y_3 = 20\)

Budget constraint: \[
c_1 + c_2 + c_3 = 16 + 20 + 20 = 56
\]

Maximize utility: \[
U = u(c_1) + u(c_2) + u(c_3) = \sum_{t=1}^3 \left(c_t - \frac{1}{100}c_t^2\right)
\]

This is a concave utility function, symmetric across periods, and
\textbf{no interest rate or discounting} ⇒ optimal consumption is equal
across periods:

Let \(c_1 = c_2 = c_3 = c\), then: \[
3c = 56 \Rightarrow c = \frac{56}{3} \approx 18.67
\]

\textbf{Answer:} - \(c_1 = c_2 = c_3 \approx 18.67\)

\subsubsection{Government announces in period 1 that a 20\% income tax
will be levied in periods 2 and
3}\label{government-announces-in-period-1-that-a-20-income-tax-will-be-levied-in-periods-2-and-3}

Adjusted income: - \(y_1 = 20\) -
\(y_2 = y_3 = 20 \times (1 - 0.2) = 16\)

Budget constraint: \[
c_1 + c_2 + c_3 = 20 + 16 + 16 = 52
\]

Again, utility is symmetric, so optimal consumption is equal across all
periods:

\[
3c = 52 \Rightarrow c = \frac{52}{3} \approx 17.33
\]

\textbf{Answer:} - \(c_1 = c_2 = c_3 \approx 17.33\)

\subsubsection{No announcement, tax 20\% unexpectedly applied in periods
2 and
3}\label{no-announcement-tax-20-unexpectedly-applied-in-periods-2-and-3}

In this case, the consumer \textbf{does not anticipate the tax} when
choosing \(c_1\). So she expects full income in all periods: - Expected
income: \(y_1 = y_2 = y_3 = 20 \Rightarrow\) total = 60 - Optimal
planned consumption: \(c_1 = c_2 = c_3 = 60 / 3 = 20\)

But in periods 2 and 3, actual income is only \(16\) each (due to tax),
so consumer faces a \textbf{budget shortfall}:

\begin{itemize}
\tightlist
\item
  \(c_1 = 20\)
\item
  Income in t=2,3 is 16; consumption was planned as 20 ⇒ needs to adjust
\end{itemize}

Let's assume the consumer \textbf{re-optimizes consumption} in periods 2
and 3 under new constraint:

Remaining resources in t=2,3: \(16 + 16 = 32\)

She wants to equalize consumption in periods 2 and 3:

\[
2c = 32 \Rightarrow c = 16
\]

So actual consumption becomes: - \(c_1 = 20\) - \(c_2 = c_3 = 16\)

\textbf{Answer:} - \(c_1 = 20\) - \(c_2 = c_3 = 16\)

\textbf{Summary Table}

\begin{longtable}[]{@{}
  >{\raggedright\arraybackslash}p{(\linewidth - 6\tabcolsep) * \real{0.6489}}
  >{\raggedleft\arraybackslash}p{(\linewidth - 6\tabcolsep) * \real{0.1170}}
  >{\raggedleft\arraybackslash}p{(\linewidth - 6\tabcolsep) * \real{0.1170}}
  >{\raggedleft\arraybackslash}p{(\linewidth - 6\tabcolsep) * \real{0.1170}}@{}}
\toprule\noalign{}
\begin{minipage}[b]{\linewidth}\raggedright
Scenario
\end{minipage} & \begin{minipage}[b]{\linewidth}\raggedleft
\(c_1\)
\end{minipage} & \begin{minipage}[b]{\linewidth}\raggedleft
\(c_2\)
\end{minipage} & \begin{minipage}[b]{\linewidth}\raggedleft
\(c_3\)
\end{minipage} \\
\midrule\noalign{}
\endhead
\bottomrule\noalign{}
\endlastfoot
(1) Given: \(c_1=15, c_2=20, c_3=20\) & 15 & 20 & 20 \\
(2) Tax in period 1 only & ≈18.67 & ≈18.67 & ≈18.67 \\
(3) Tax announced in t=1 for t=2 and t=3 & ≈17.33 & ≈17.33 & ≈17.33 \\
(4) Tax applied in t=2 and t=3 unexpectedly (no announcement) & 20 & 16
& 16 \\
\end{longtable}

\begin{center}\rule{0.5\linewidth}{0.5pt}\end{center}

\subsection{Task 3: Data Economy}\label{task-3-data-economy}

We are given an economy with \textbf{two identical firms} using the
production technology: \[
Y_i = A \cdot K_i, \quad \text{for } i = 1, 2
\]

\textbf{Common Technology:} Both firms share a common productivity
factor: \[
A = d^\eta = (0.5Y_1 + 0.5Y_2)^\eta, \quad \eta = 0.5
\]

Since \(Y_i = A \cdot K_i\), we substitute: \[
A = \left(0.5AK_1 + 0.5AK_2\right)^\eta = \left(0.5A(K_1 + K_2)\right)^\eta
\]

Let's denote \(K = K_1 + K_2\), then: \[
A = \left(0.5A \cdot K\right)^\eta
\]

\subsubsection{\texorpdfstring{Express \(A\) as a function of \(K_1\)
and
\(K_2\)}{Express A as a function of K\_1 and K\_2}}\label{express-a-as-a-function-of-k_1-and-k_2}

From: \[
A = (0.5A(K_1 + K_2))^\eta
\]

Raise both sides to the power \(1/\eta\): \[
A^{1/\eta} = 0.5A(K_1 + K_2)
\]

Substitute \(\eta = 0.5 \Rightarrow 1/\eta = 2\): \[
A^2 = 0.5A(K_1 + K_2)
\]

Divide both sides by \(A\) (assuming \(A > 0\)): \[
A = 0.5(K_1 + K_2)
\]

\textbf{Answer (1.3.1):} \[
A = 0.5(K_1 + K_2)
\]

\subsubsection{Optimization Problem}\label{optimization-problem}

Firm \(i\) chooses \(K_i\) to maximize: \[
\pi_i = A \cdot K_i - R \cdot K_i
\]

From (1), recall: \[
A = 0.5(K_1 + K_2)
\]

Substitute into the profit function: \[
\pi_1 = 0.5(K_1 + K_2) \cdot K_1 - 0.2 \cdot K_1
\]

Take first-order condition (FOC) with respect to \(K_1\): \[
\frac{d\pi_1}{dK_1} = 0.5K_1 + 0.5K_2 - 0.2 = 0
\Rightarrow K_1 + K_2 = 0.4
\]

By symmetry (identical firms): \(K_1 = K_2 = K\)

So: \[
2K = 0.4 \Rightarrow K = 0.2
\]

\textbf{Answer (1.3.2):} \[
K_1 = K_2 = 0.2
\]

\textbf{Final Summary}

\begin{itemize}
\tightlist
\item
  \begin{enumerate}
  \def\labelenumi{(\arabic{enumi})}
  \tightlist
  \item
    Productivity is: \[
    A = 0.5(K_1 + K_2)
    \]
  \end{enumerate}
\item
  \begin{enumerate}
  \def\labelenumi{(\arabic{enumi})}
  \setcounter{enumi}{1}
  \tightlist
  \item
    Optimal capital allocations: \[
    K_1 = K_2 = 0.2, \quad A = 0.5(0.2 + 0.2) = 0.2
    \]
  \end{enumerate}
\end{itemize}

Profit for each firm: \[
\pi_i = A \cdot K_i - R \cdot K_i = 0.2 \cdot 0.2 - 0.2 \cdot 0.2 = 0
\]

Each firm earns \textbf{zero profit} at the optimum under this
technology specification and capital cost.

\begin{center}\rule{0.5\linewidth}{0.5pt}\end{center}

\end{document}
